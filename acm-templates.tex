\documentclass[a4paper,11pt]{article}
\usepackage[UTF8]{ctex}
\usepackage{pdfpages}
\usepackage{zh_CN-Adobefonts_external}
\usepackage{geometry}
\usepackage{fancyhdr}
\usepackage{minted}
\usepackage[colorlinks,
linkcolor=black,CJKbookmarks,
]{hyperref}
\usepackage[T1]{fontenc}
\usepackage[utf8]{inputenc}
\setlength{\headheight}{15pt}
\pagestyle{fancy}
\fancyhf{}
\fancyhead[C]{ACM Templates by XTS}
\lfoot{}
\cfoot{\thepage}
\rfoot{}

\author{XTS}
\title{ACM Templates}
\hyphenpenalty=500000
\tolerance=10
\geometry{a4paper,left=2cm,right=2cm,top=3cm,bottom=3cm}
\begin{document}
    \maketitle
    \newpage
    \tableofcontents




    \newpage
    \section{图论} % 一级标题

    \subsection{dijstra}
    \inputminted[breaklines]{c++}{Graph/dijstra.cpp}

    \subsection{倍增LCA}%三级标题
    \inputminted[breaklines]{c++}{Graph/lca.cpp}

    \subsection{树上路径交}
    \inputminted[breaklines]{c++}{Graph/树上路径交.cpp}

    \subsection{树链剖分}
    \inputminted[breaklines]{c++}{Graph/树链剖分.cpp}

    \subsection{点分治}
    \inputminted[breaklines]{c++}{Graph/点分治.cpp}

    \subsection{强连通分量}
    \inputminted[breaklines]{c++}{Graph/scc.cpp}

    \subsection{双连通分量}
    \inputminted[breaklines]{c++}{Graph/bcc.cpp}

    \subsection{仙人掌求环}
    \inputminted[breaklines]{c++}{Graph/仙人掌求环.cpp}

    \subsection{dfs判负环}
    \inputminted[breaklines]{c++}{Graph/dfs判负环.cpp}

    \subsection{斯坦纳树}
    \inputminted[breaklines]{c++}{Graph/斯坦纳树.cpp}

    \subsection{二分图最大权匹配}
    \inputminted[breaklines]{c++}{Graph/KM.cpp}

    \subsection{最大流}
    \inputminted[breaklines]{c++}{Graph/Dinic.cpp}

    \subsection{最小割树}
    \inputminted[breaklines]{c++}{Graph/最小割树.cpp}
    
    \subsection{网络流的一些结论}
	\begin{document}
		fsa\\dasdsa\\\sdadsad\\
		
	\end{document}
    %\twocolumn  分栏






    \newpage
    \section{字符串}

    \subsection{回文树}
    \inputminted[breaklines]{c++}{String/pam.cpp}

    \subsection{广义后缀自动机}
    \inputminted[breaklines]{c++}{String/sam.cpp}

    \subsection{kmp}
    \inputminted[breaklines]{c++}{String/kmp.cpp}


    \newpage
    \section{数据结构}











    \newpage
    \section{数学} % 一级标题

    \subsection{带预处理BGSG}
    \inputminted[breaklines]{c++}{Math/BGSG.cpp}

    \subsection{十进制矩阵快速幂}
    \inputminted[breaklines]{c++}{Math/快速幂.cpp}

    \subsection{exlucas}
    \inputminted[breaklines]{c++}{Math/exlucas.cpp}

    \subsection{BM}
    \inputminted[breaklines]{c++}{Math/BM.cpp}

    \subsection{杜教筛}
    \inputminted[breaklines]{c++}{Math/杜教筛.cpp}

    \subsection{线性基}
    \inputminted[breaklines]{c++}{Math/线性基.cpp}

    \subsection{原根}
    \includepdf[pages={1}]{Math/原根.pdf}

    \subsection{欧拉降幂}
    \inputminted[breaklines]{c++}{Math/欧拉降幂.cpp}

    \subsection{拉格朗日插值}
    \includepdf[pages={1,2}]{Math/拉格朗日插值.pdf}

    \subsection{伯努利数}
    \includepdf[pages={1,2}]{Math/伯努利数.pdf}

    \subsection{FFTNTTFWT}
    \includepdf[pages={1,2,3,4,5,6}]{Math/FFTNTTFWT.pdf}


    \newpage
    \section{杂项}

    \subsection{约瑟夫问题}
    \includepdf[pages={1,2,3}]{Others/约瑟夫问题.pdf}

    \subsection{RSA}
    \includepdf[pages={1}]{Others/RSA.pdf}

    \subsection{快速读}
    \inputminted[breaklines]{c++}{Others/quick_IO.cpp}

    \subsection{并没有任何用处的卡常小知识}
    \inputminted[breaklines]{c++}{Others/并没有任何用处的卡常小知识.cpp}

    \subsection{杂杂杂项}
    \includepdf[pages={1,2,3,4,5,6,7}]{Others/推导用.pdf}

    \subsection{pbds}
    \includepdf[pages={1-14}]{Others/pbds.pdf}





    \newpage
    \section{计算几何}


    %\newpage
    %\section{Others}

\end{document}


